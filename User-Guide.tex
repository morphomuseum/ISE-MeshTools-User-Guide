%Préambule du document :
\documentclass[12pt]{book}
%\usepackage[latin1]{inputenc} 

\usepackage{graphicx}
\usepackage{titling}

% set up \maketitle to accept a new item
\predate{\begin{center}\placetitlepicture\large}
\postdate{\par\end{center}}

% commands for including the picture
\newcommand{\titlepicture}[2][]{%
  \renewcommand\placetitlepicture{%
    \includegraphics[#1]{#2}\par\medskip
  }%
}
\newcommand{\placetitlepicture}{} % initialization

\title{ISE-MeshTools User's guide\\ISE-MeshTools v. 1.3}

%\titlepicture[width=15cm]{header.png}
\author{Renaud LEBRUN}
\date{28th february 2016} 


%Corps du document :
\begin{document}



	\maketitle
   ISE-MeshTools is a software designed by Renaud Lebrun, from the university of Montpellier II. ISE-MeshTools is a 
system for the processing and editing of series of 3D triangular meshes. The system provides a set of tools for editing, 
positioning, deforming, labelling, measuring and rendering sets of 3D meshes. Features include:\\
•Retrodeformation for un-deforming fossils/deformed specimens\\
•Point and curve primitives for placing the exact type of landmark points you’re interested in\\
•Easy to use 3D interface for positioning and manipulating sets of surfaces and landmark primitives\\
•Mesh tagging, labelling and colouring (to allow for the creation of anatomy atlases)\\
• Mesh scalar computation and colouring (based upon curvature/thickness etc...)

		 \input{chapter_01_licence.tex}
		
 \chapter{F.A.Q.}
		\minitoc  
    \section{How should I cite ISE-MeshTools in scientific publications ?}
    You may  cite ISE-MeshTools with the following reference :\\
		Lebrun, R. ISE-MeshTools, a 3D interactive fossil reconstruction freeware. 
		12th Annual Meeting of EAVP, Torino, Italy; 06/2014.
    \section{Is ISE-MeshTools a geometric morphometrics software ?}
    No. However, you can digitize 3D landmarks on complex 3D surfaces using ISE-MeshTools, which you 
		can use in other software.
		\section{Can I produce/extract 3D meshes out of CT/MRI data using ISE-MeshTools ?}
		No. To extract 3D meshes CT/MRI data sets, you have to use another software. However, you can edit 
		3D Meshes in various ways using ISE-MeshTools.
		\section{Is there a CTRL-Z functionnality around ?}
		No, there is currently no way to cancel any action. So remember to regularly save your work (especially when tagging surfaces), otherwise precious hours of work can be lost in a second.
   
		
\end{document} 
