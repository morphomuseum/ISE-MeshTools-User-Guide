%Préambule du document :
\documentclass[12pt]{book}
%\usepackage[latin1]{inputenc} 

\usepackage{graphicx}
\usepackage{titling}

% set up \maketitle to accept a new item
\predate{\begin{center}\placetitlepicture\large}
\postdate{\par\end{center}}

% commands for including the picture
\newcommand{\titlepicture}[2][]{%
  \renewcommand\placetitlepicture{%
    \includegraphics[#1]{#2}\par\medskip
  }%
}
\newcommand{\placetitlepicture}{} % initialization

\title{ISE-MeshTools User's guide\\ISE-MeshTools v. 1.3}

%\titlepicture[width=15cm]{header.png}
\author{Renaud LEBRUN}
\date{28th february 2016} 


%Corps du document :
\begin{document}



	\maketitle
   ISE-MeshTools is a software designed by Renaud Lebrun, from the university of Montpellier II. ISE-MeshTools is a 
system for the processing and editing of series of 3D triangular meshes. The system provides a set of tools for editing, 
positioning, deforming, labelling, measuring and rendering sets of 3D meshes. Features include:\\
•Retrodeformation for un-deforming fossils/deformed specimens\\
•Point and curve primitives for placing the exact type of landmark points you’re interested in\\
•Easy to use 3D interface for positioning and manipulating sets of surfaces and landmark primitives\\
•Mesh tagging, labelling and colouring (to allow for the creation of anatomy atlases)\\
• Mesh scalar computation and colouring (based upon curvature/thickness etc...)

		 \chapter{Licence}
    
		\minitoc
		
    \section{ISE-MeshTools}
    ISE-MeshTools is Copyright(C) 2013-2016: Renaud LEBRUN, Cécile PELADAN, Stefan SCHLAGER, Jean DUMONCEL. All rights reserved.
This program is free software; you can redistribute it and/or modify it under the terms of the GNU 
General Public License as published by the Free Software Foundation; either version 2 of the License, 
or any later version.\\\\
This program is distributed in the hope that it will be useful, but WITHOUT ANY WARRANTY; without 
even the implied warranty of MERCHANTABILITY or FITNESS FOR A PARTICULAR PURPOSE. See the 
GNU General Public License for more details.

    \section{VTK}
  ISE-MeshTools’ compiled versions contain binary forms of VTK: Copyright (c) 2000-2006 Kitware Inc. 28 
Corporate Drive, Suite 204, Clifton Park, NY, 12065, USA. All rights reserved. Redistribution and use 
in source and binary forms, with or without modification, are permitted provided that the following 
conditions are met:\\
    Redistributions of source code must retain the above copyright notice, this list of conditions and 
the following disclaimer.\\
    Redistributions in binary form must reproduce the above copyright notice, this list of conditions 
and the following disclaimer in the documentation and/or other materials provided with the distribution.//
    Neither the name of Kitware nor the names of any contributors may be used to endorse or promote products derived from this software without specific prior written permission.\\\\
THIS SOFTWARE IS PROVIDED BY THE COPYRIGHT HOLDERS AND CONTRIBUTORS ``AS IS’’ AND ANY 
EXPRESS OR IMPLIED WARRANTIES, INCLUDING, BUT NOT LIMITED TO, THE IMPLIED WARRANTIES 
OF MERCHANTABILITY AND FITNESS FOR A PARTICULAR PURPOSE ARE DISCLAIMED. IN NO EVENT 
SHALL THE AUTHORS OR CONTRIBUTORS BE LIABLE FOR ANY DIRECT, INDIRECT, INCIDENTAL, SPECIAL, 
EXEMPLARY, OR CONSEQUENTIAL DAMAGES (INCLUDING, BUT NOT LIMITED TO, PROCUREMENT OF 
SUBSTITUTE GOODS OR SERVICES; LOSS OF USE, DATA, OR PROFITS; OR BUSINESS INTERRUPTION) 
HOWEVER CAUSED AND ON ANY THEORY OF LIABILITY, WHETHER IN CONTRACT, STRICT LIABILITY, 
OR TORT (INCLUDING NEGLIGENCE OR OTHERWISE) ARISING IN ANY WAY OUT OF THE USE OF THIS 
SOFTWARE, EVEN IF ADVISED OF THE POSSIBILITY OF SUCH DAMAGE.

		 \chapter{F.A.Q.}
  
    \section{How should I cite ISE-MeshTools in scientific publications ?}
    You may  cite ISE-MeshTools with the following reference :\\
		Lebrun, R. ISE-MeshTools, a 3D interactive fossil reconstruction freeware. 
		12th Annual Meeting of EAVP, Torino, Italy; 06/2014.
    \section{Is ISE-MeshTools a geometric morphometrics software ?}
    No. However, you can digitize 3D landmarks on complex 3D surfaces using ISE-MeshTools, which you 
		can use in other software.
		\section{Can I produce/extract 3D meshes out of CT/MRI data using ISE-MeshTools ?}
		No. To extract 3D meshes CT/MRI data sets, you have to use another software. However, you can edit 
		3D Meshes in various ways using ISE-MeshTools.
		\section{Is there a CTRL-Z functionnality around ?}
		No, there is currently no way to cancel any action. So remember to regularly save your work (especially when tagging surfaces), otherwise precious hours of work can be lost in a second.
   
		
\end{document} 
